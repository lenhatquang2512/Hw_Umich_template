\documentclass[a4paper, 12pt]{article}
\usepackage[left=1in,top=1in,right=1in,bottom=1in]{geometry}
% \usepackage[margin=1in]{geometry}
% \renewcommand{\baselinestretch}{1.2}
% \setlength{\emergencystretch}{10pt}
\setlength{\parindent}{20pt}

\usepackage{algorithm,algpseudocode}
\usepackage{tabularx}
\usepackage[table]{xcolor}
\usepackage{amsmath, amssymb, amsthm, amscd}
\usepackage[rightcaption]{sidecap}
\usepackage{derivative}
\usepackage{wrapfig}
\usepackage{fancyhdr}
\usepackage{listings}
\usepackage{enumitem}% http://ctan.org/pkg/enumitem
\usepackage{pythonhighlight}
\usepackage{graphicx} %package to manage images
\graphicspath{ {./images/} }
\usepackage{pdfpages}
\usepackage{verbatim}
\usepackage{setspace}
\usepackage{hyperref} % create hyper link
\usepackage{multicol,multirow}  %% Figure in two columns
\usepackage{bm}
\usepackage{mathtools, nccmath}

\pagestyle{fancy}
\fancyhf{}
\rhead{Quang Nhat Le}
\cfoot{\thepage}

%setting to eliminate the warning
\setlength{\headheight}{15pt}

%change the size of title header
\usepackage[small]{titlesec}

% color  MATLAB
% \usepackage{color} %red, green, blue, yellow, cyan, magenta, black, white 
% \definecolor{mygreen}{RGB}{28,172,0} % color values Red, Green, Blue
% \definecolor{mylilas}{RGB}{170,55,241}

% color  PYTHON
\usepackage{color}

%reference part
\usepackage{cleveref}
% \usepackage{biblatex} %Imports biblatex package
\usepackage[sorting=none]{biblatex}
\addbibresource{mybib.bib} %Import the bibliography file

%set up line spacing for the whole document
% \doublespacing
% \onehalfspacing
\singlespacing
% To make a part of the text of your document singlespaced, you can put:
% \begin{singlespace}
% at the beginning of the text you want singlespaced, and
% \end{singlespace}


\begin{document}

\title{\textbf{EECS 498 - HW5}}
\author{\textbf{Quang Le} \\ \textit{quangle@umich.edu}}
\date{}
% 	\vspace{-3cm}
\maketitle

\begin{sloppypar}
% \noindent

%-----------------------------PYTHON setting--------------------------------------------

%%
%%  Define some colours trying to mimic TigerJython. 
%%
\definecolor{keywordclr}{rgb}{0, 0.2, 0.667}      
\definecolor{fnctionclr}{rgb}{0.467, 0, 0.533}
\definecolor{builtinclr}{rgb}{0.467, 0, 0.533}
\definecolor{symbolsclr}{rgb}{0.5, 0.25, 0.25}   
\definecolor{commentclr}{rgb}{0, 0.5, 0}       
\definecolor{stringsclr}{rgb}{0.8, 0.4, 0}     
\definecolor{numbersclr}{rgb}{0.8, 0.2, 0}
\definecolor{bckgrndclr}{rgb}{0.91, 0.95, 0.95}
%%
%%  Define a new language 'Jython'. We also want strings over multiple lines 
%%  to be formatted as strings and not as comments.
%%
\lstdefinelanguage{Jython}%
  { morekeywords={None,and,as,assert,break,class,continue,def,del,elif,%
      else,except,finally,for,from,global,if,import,in,is,lambda,nonlocal,%
      not,or,pass,print,raise,return,try,while,with,yield,repeat},%
    moreprocnamekeys={class,def},%
    %%  We use 'classoffset=1' to add built in functions (__builtins__)
    classoffset=1,%
    morekeywords={Exception,False,True,abs,all,any,ascii,bin,bool,bytearray,%
      bytes,callable,chr,classmethod,compile,complex,copyright,delattr,dict,%
      dir,divmod,enumerate,eval,exec,exit,float,format,frozenset,getattr,%
      globals,hasattr,hash,help,hex,id,input,int,isinstance,issubclass,iter,%
      len,license,list,locals,map,max,min,next,object,oct,open,ord,pow,%
      property,quit,range,repr,reversed,round,set,setattr,slice,sorted,%
      staticmethod,str,sum,super,tuple,type,vars,zip,inputInt,inputFloat,%
      inputString,msgDlg,askYesNo,enum,clrScr,head,tail,indices,self},%
    classoffset=0,%
    sensitive=true,%
    showstringspaces=false,%
    morecomment=[l]\#,%
    morestring=[b]',%
    morestring=[b]",%
    morestring=[b]{'''},%
    morestring=[b]{"""}%
  }
%%
%%  Set 'Jython' as default language and set all the colors and styles.
%%
% \lstset{language=Jython,% 
%   extendedchars=true,%
%   belowskip=0pt,%
%   framexleftmargin=2pt,%
%   framexrightmargin=2pt,%
%   numbers=none,%
%   basicstyle=\ttfamily\small,%
%   numberstyle=\scriptsize,%
%   keywordstyle=\color{keywordclr}\bfseries,% 
%   stringstyle=\color{stringsclr},%
%   commentstyle=\color{commentclr}\itshape,%
%   procnamestyle=\color{fnctionclr},%
%   tabsize=4,%
%   classoffset=1,%
%   keywordstyle=\color{builtinclr},%
%   classoffset=0,
%   backgroundcolor=\color{bckgrndclr}%
% }

%-------------------------------MATLAB setting----------------------------------------------

% \lstset{
%     % language=Matlab,%
%     %basicstyle=\color{red},
%     breaklines=true,%
%     morekeywords={matlab2tikz},
%     keywordstyle=\color{blue},%
%     morekeywords=[2]{1}, keywordstyle=[2]{\color{black}},
%     identifierstyle=\color{black},%
%     stringstyle=\color{mylilas},
%     commentstyle=\color{mygreen},%
%     showstringspaces=false,%without this there will be a symbol in the places where there is a space
%     numbers=left,%
%     numberstyle={\tiny \color{black}},% size of the numbers
%     numbersep=9pt, % this defines how far the numbers are from the text
%     emph=[1]{for,end,break},emphstyle=[1]\color{red}, %some words to emphasise
%     %emph=[2]{word1,word2}, emphstyle=[2]{style},    
% }

%---------------------------------------------------------------------

%----------------------------------------------------------------------------
% \includegraphics[width=5cm, height=4cm]{overleaf-logo}
%----------------------------------------------------------------------------
% \vspace{1.5cm}
%---------------------------------------------------------------------------
% \includegraphics[scale=1.2, angle=45]{overleaf-logo}
%----------------------------------------------------------------------------
% \includegraphics[width=\textwidth]{universe}
%---------------------------------------------------------------------
%Example, floating pictures. Text wrapping the figure
% \begin{wrapfigure}{r}{0.25\textwidth} %this figure will be at the right
%     \centering
%     \includegraphics[width=0.25\textwidth]{mesh}
% \end{wrapfigure}

% \begin{wrapfigure}{l}{0.25\textwidth} %this figure will be at the left
%     \centering
%     \includegraphics[width=0.25\textwidth]{contour}
% \end{wrapfigure}

% %----------------------------------------------------------------------------
% %Example of caption next to the figure
% \begin{SCfigure}[0.5][h]
% \caption{Using the picture of the universe again. 
%          This caption will be on the right}
% \includegraphics[width=0.6\textwidth]{universe}
% \end{SCfigure}
% %----------------------------------------------------------------------------

% %----------------------------------------------------------------------------
% %Labeling and referencing example
% \begin{figure}[h]
%     \centering
%     \includegraphics[width=0.25\textwidth]{mesh}
%     \caption{a nice plot}
%     \label{fig:mesh1}
% \end{figure}

% As you can see in the figure \ref{fig:mesh1}, the function grows near 0. Also, in the page \pageref{fig:mesh1} is the same example.
% %----------------------------------------------------------------------------

% Chane to \section* if dont want numbering
% \section*{Problem 1}
\section{Problem 1}


\subsection{Part a}
\subsection{Part b}
\subsection{Part c}
\subsection{Part d}

% Adding PDF pages
% \includepdf[pages={1,3,5-7}]{mypdf.pdf}
% \includepdf[pages={1-3,{},8,10-12}]{mypdf.pdf}


% ADDING Python/ MATLAB code
% \lstinputlisting[language=Matlab]{ex2_hw8_me564.m}
% \lstinputlisting[language=Matlab, firstline=1, lastline=28]{ex1_hw10_rob501.m}
% \lstinputlisting[language=Python, firstline=1, lastline=28]{pca.py}
% \lstinputlisting{pca.py}

\begin{python}
import numpy as np
import matplotlib.pyplot as plt
import time

def train_PCA(train_data):
  ##### TODO: Implement here!! #####
  # Note: do NOT use sklearn here!
  # Hint: np.linalg.eig() might be useful
  states = {
      'transform_matrix': np.identity(train_data.shape[-1]),
      'eigen_vals': np.ones(train_data.shape[-1])
  }
  ##### TODO: Implement here!! #####
  return states

# Load data
start = time.time()
images = np.load('pca_data.npy')
num_data = images.shape[0]
train_data = images.reshape(num_data, -1)

\end{python}


\begin{verbatim}
The Jacobian of f is : 
[- 3*x3^3 + 6*x2, (4*x2^3)/3 + 6*x1, -9*x1*x3^2]
 
Evaluate Jacobian of f at xs is : 
[21, 42, -9]
\end{verbatim}


% \begin{figure}[h!]
%     \centering
%     \includegraphics[scale = 0.6]{}
%     \caption{}
%     % \label{fig:mesh0}
% \end{figure}



% \begin{figure}[h!]
%     \centering
%     \includegraphics[width=1.0\linewidth]{}
%     \caption{}
%     % \label{fig:mesh0}
% \end{figure}




% \newpage
\clearpage
\section{Problem 2}

\subsection{Part a}
\subsection{Part b}
\subsection{Part c}
% \subsubsection*{Part c(ii)}



% \newpage
\clearpage
\section{Problem 3}

\subsection{Part a}
\subsection{Part b}
\subsection{Part c}
% \subsubsection*{Part c(ii)}






% %-------------------------- MORE TEMPLATE--------------------------------------------------
% \subsection{Part a}
% \subsection{Part b}
% \subsection{Part c}

% Adding PDF pages
% \includepdf[pages={1,3,5-7}]{mypdf.pdf}
% \includepdf[pages={2-3,{},4}]{ME564HW9_1-9wocode.pdf}
% \includepdf[pages={1-3,{},8,10-12}]{mypdf.pdf}

% no need caption
% \lstinputlisting[language=Matlab]{ex2_hw8_me564.m}
% \lstinputlisting[language=Matlab, firstline=1, lastline=28]{ex1_hw10_rob501.m}
% \lstinputlisting[language=Python, firstline=1, lastline=28]{pca.py}
% \lstinputlisting{pca.py}

% INSERTING PYTHON CODES
% \begin{python}

% \end{python}

% INSERTING OUTPUT TERMINAL
% \begin{verbatim}

% \end{verbatim}


% \begin{figure}[h!]
%     \centering
%     \includegraphics[scale = 0.6]{ex3a_linear.png}
%     \caption{3a code - LS }
%     % \label{fig:mesh0}
% \end{figure}

% \begin{figure}[h!]
%     \centering
%     \includegraphics[width=1.0\linewidth]{c-i-random.png}
%     \caption{Final value using random initialization}
%     % \label{fig:mesh0}
% \end{figure}

% \begin{equation}
%     m = K * (d+1)
% \end{equation}

\begin{equation}
    \label{lowbound}
    \underline{P}(x_q) = \min_{P(X_i|\pi_i) \in K(X_i|\pi_i)} \sum_{x_1,...,x_n \backslash x_q} \prod_{i=0}^n P(x_i|\pi_i)
\end{equation}

\begin{equation}
    \label{Chap3:Eq2}
    \begin{split}
        \min_{\alpha}\quad & \frac{1}{2}{\alpha}^{T}Q \alpha - e^T\alpha \\
        \text{subject to } \quad & y^T \alpha = 0 \\
        \text{with } \quad &  0 \leq {\alpha}_i \leq C, i = \{ 1,2,..,62\}
    \end{split}
\end{equation}

\begin{align}
    \min_{\alpha}\quad      & \frac{1}{2}{\alpha}^{T}Q \alpha - e^T\alpha \\
    \text{subject to }\quad & y^T \alpha = 0                              \\
    \text{with }  \quad     & 0 \leq {\alpha}_i \leq C, i = \{ 1,2 \}
\end{align}

\begin{equation} \renewcommand{\arraystretch}{1.8}
    \label{Chap3:Eq4}
    \begin{pmatrix}
        v_t \\ \omega_{t}
    \end{pmatrix}=
    \begin{pmatrix}
        \dot{d}(t) \\ \dot{\theta} (t)
    \end{pmatrix}= K_p \begin{pmatrix}
        e_d (t) \\e_{\theta} (t)
    \end{pmatrix} + K_i \int_{0}^{T} \begin{pmatrix}
        e_d (t) \\e_{\theta} (t)
    \end{pmatrix} \mathrm{dt} + K_d \frac{d}{\mathrm{dt}} \begin{pmatrix}
        e_d (t) \\e_{\theta} (t)
    \end{pmatrix}
\end{equation}

% \begin{enumerate}[label={(\arabic*)}]
%   \item \textbf{How to compare 2 BIC values} \\
 
  
%   \item \textbf{How to choose best number of clusters given multiple runs with the same K} \\
 
% \end{enumerate}


% -------------------------------TABLE-----------------------------------

Include Tables like Table \ref{tab:table}. More info on tables: 
\url{https://www.overleaf.com/learn/latex/Tables}.
\begin{table}[h]
    \centering
    \begin{tabular}{c|cc}
    \hline
    Disruption $\setminus$ $X^t$ & Available &  Unavailable \\ \hline \hline
    False      & $[0.952, 0.962]$ & $[0.038, 0.048]$ \\
    True       & $[0.038, 0.048]$ & $[0.952, 0.962]$ \\ \hline
    \end{tabular}
    \caption{This is a table} 
    \label{tab:table}
\end{table}

\begin{table}[h!]
    \centering
    \begin{tabular}{||c|c|c||}
        \hline
        \rowcolor{lightgray}
        \textbf{Feature} & \textbf{Description}                                & \textbf{Dimension} \\ [0.5ex]
        \hline\hline
        $f_1$            & Number of points in cluster                         & 1                  \\
        $f_2$            & Minimum cluster distance from LiDAR                 & 1                  \\
        $f_3$            & Cluster volume                                      & 1                  \\
        $f_4$            & 3D covariance matrix of cluster                     & 6                  \\
        $f_5$            & Normalized moment of inertia tensor                 & 6                  \\
        $f_6$            & Slice feature of the cluster                        & 20                 \\
        $f_7$            & Reflection intensity's distribution (mean, std dev) & 27                 \\ [0.5ex]
        \hline
    \end{tabular}
    \caption{All features for human classification}
    \label{Chap3:Table1}
\end{table}

\begin{table}[h!]
    \centering
    \begin{tabular}{||c|c||}
        \hline
        \rowcolor{lightgray}
        \textbf{Item}                 & \textbf{Specifications}                     \\ [0.5ex]
        \hline\hline
        Scanning rate                 & 10 scans/s                                  \\ \hline
        Horizontal field of view      & $360^o$                                     \\ \hline
        Horizontal angular resolution & $0.23^o$                                    \\ \hline
        Vertical field of view        & $26.8^o$                                    \\ \hline
        Vertical angular resolution   & $2^o$ (16 lines)                            \\ \hline
        Detection range               & 40m for pavement, 120m for cars and foliage \\ \hline
        Range accuracy                & 0.02m                                       \\ \hline
        Wavelength of laser beam      & 905nm                                       \\ [0.5ex]
        \hline
    \end{tabular}
    \caption{3D VLP-16 LiDAR properties \cite{vlp16}}
    \label{Chap4:Table1}
\end{table}


% SUB-FIGURES
% \begin{figure*}[h]
%     \centering
%     \begin{minipage}{\columnwidth}
%         \includegraphics[width=0.48\linewidth]{figures/chap2_fig/shins_human.png}
%         \hfill
%         \includegraphics[width=0.48\linewidth]{figures/chap2_fig/shinshuman.png}
%         \caption{Shins Human detection and following using LRS\cite{hfr_lrs}}
%         \label{Chap2:Fig2}
%     \end{minipage}\hfill
% \end{figure*}

% \begin{figure}[h]
%     \centering
%     \begin{subfigure}{.5\linewidth}
%         \centering
%         \includegraphics[width=1.0\linewidth,height = 0.5\linewidth]{figures/chap4_fig/Platform/lidar_property.png}
%           \caption{LiDAR property \cite{lidarproperties}}
%         \label{chap4:fig1:sub1}
%     \end{subfigure}%
%     \begin{subfigure}{.5\linewidth}
%         \centering
%         \includegraphics[width=1.0\linewidth,height = 0.5\linewidth]{figures/chap4_fig/Platform/Untitled Diagram-Page-12.drawio.png}
%           \caption{Model calculate height of LiDAR \cite{calculateheight}}
%         \label{chap4:fig1:sub2}
%     \end{subfigure}
%     \caption{VLP-16 LiDAR vertical field of view}
%     \label{chap4:fig1}
% \end{figure}


% ----------------------ALGORITHM-------------------------------
\begin{algorithm}[h]
    \caption{RANSAC(Random Sample Consensus) ground plane detection algorithm  }
    \label{Chap3:Alg1}
    \begin{algorithmic}[1]
        \Ensure Point cloud data $P^*$ of the plane model
        \Require Point cloud data $P$, where $p_i=[x,y,z] \in P$;
        \State Randomly select three non-collinear unique points $\{p_i, p_j, p_k\}$ from $P$;
        \State Compute the model coefficients from the three points $(ax + by + cz + d = 0)$;
        \State Compute the distances from all $p \in P^* \subset P$ to the plane model $(a,b,c,d)$;
        \State Count the number of points $p^* \in P$ whose distance $d$ to the plane model falls between $0 \leq |d| \leq |d_t|$, where $d_t$ represents a user specified threshold.
    \end{algorithmic}
\end{algorithm}


\begin{algorithm}[h]
    \caption{Euclidean clustering algorithm  \cite{rusu_thesis,cnn_uav}}
    \label{Chap3:Alg2}
    \begin{algorithmic}[1]
        \Ensure List of point cloud clusters C
        \Require Point cloud data $P$, where $p_i=[x,y,z] \in P$;
        \State Create a Kd-tree representation for $P$;
        \State Setup an empty list of clusters $C$, and a queue of points to be checked $Q$;
        \For{every point $p_i \in P$}
        \State Add $p_i$ to the current queue $Q$;
        \For {every point $p_i \in Q$}
        \State Search for the set ${p_i}^k$ of point neighbours of $p_i$ s.t a sphere with radius $r,d^*$
        \State For every neighbour ${p_i}^k \in {p_i}^k$ check if the point has already been processed, else add it to Q;
        \EndFor
        \State When the list of all points in $Q$ has been processed, add $Q$ to the list of clusters $C$, and $Q \gets \emptyset$
        \EndFor
        \State Return the algorithm terminates when all point $\in P$ have been processed;
    \end{algorithmic}
\end{algorithm}



% %----------------------------------------------------------------------------
% \begin{figure}[h!]
%     \centering
%     \includegraphics[width=1.0\linewidth, height = 3.2 in]{6b_d2.jpg}
%     \caption{Response of linear and non-linear model of $d2$}
%     \label{fig:mesh4}
% \end{figure}

% \begin{figure}[h]
%     \centering
%     \hspace*{2cm}
%     \includegraphics[scale=0.8]{figures/chap3_fig/pid/pid_framework_2.png}
%     \caption{Human target distance and direction of mobile robot \cite{hfr_lrs}}
%     \label{Chap3:Fig20}
% \end{figure}



%----------------------------------------------------------------------------
%Generating a list of figures

% \listoffigures
% \lstlistoflistings

%------------------------------References------------------------------------------

\clearpage
\nocite{*}
% \printbibliography[heading=none]
\printbibliography


\end{sloppypar}
\end{document}


